\chapter{Introduction}
This chapter will introduce the study, which will address the issue of motorcycle accidents that will be caused by riders who will not wear helmets. It will outline the proposed Helmet Compliance Detection Using Computer Vision for Safer Roads, along with its objectives, significance, scope, and key terms.
\begin{refsection}
\section{Background of the Problem}
Motorbike accidents have been steadily increasing worldwide, leading to severe injuries and fatalities. One major contributing factor is the lack of helmet compliance and the dangerous practice of triple riding. In India alone, over 37 million individuals own and operate two-wheelers, making it critical to implement an effective monitoring system to enforce safety regulations and reduce accidents. A webcam is used for real-time video input, capturing and processing images to detect violations. The trained neural network then analyzes the webcam input, providing output based on the learned data. The system achieves an estimated 70\% accuracy, with future improvements aimed at enhancing detection precision and real-time performance.\cite{Maddi2023}. Many motorcyclists frequently violate traffic rules by not wearing helmets, and enforcement by traffic police is often limited due to the demanding nature of manual monitoring. This automated helmet detection prototype has the potential to enhance traffic law enforcement and reduce human intervention, leading to safer road environments \cite{Godbole2024}. 

The requirement for ongoing surveillance, particularly in busy locations or along lengthy stretches of road, exacerbates this problem. The safety of motorcycle riders is directly put at risk by the ineffectiveness of the enforcement procedures. The creation of an automated, vision-based safety identification and monitoring that can precisely identify the presence or absence of helmets in real-time is required to solve this issue \cite {Kumar2023}. Given the significant portion of traffic-related fatalities attributed to motorcycle accidents resulting from non-compliance with helmet regulations. Acknowledging the critical role of helmets in rider protection, this paper presents an innovative approach to helmet violation detection using deep learning methodologies \cite{Said2024}. 

Deep learning is a subset of machine learning that uses artificial neural networks to learn from large amounts of data. In automatic helmet detection, deep learning models are trained using large datasets of helmet-wearing and non-helmet-wearing people. The neural networks learn to recognize the features that distinguish helmet-wearing one from non-helmet-wearing one . Once trained, the deep learning model can be used to automatically detect whether one is wearing a helmet or not \cite{Thakur2024}.

Motorcycle-related accidents have become a growing concern worldwide, significantly contributing to road injuries and fatalities. According to the World Health Organization (WHO), more than 1.35 million people die annually due to road crashes, with motorcycle riders being among the most vulnerable. One of the leading causes of these accidents is the failure to wear helmets, which serve as a critical protective measure against head injuries. Despite laws mandating helmet use, non-compliance remains a widespread issue, exacerbated by weak enforcement and inadequate monitoring. In the Philippines, motorcycle accidents have significantly increased over the years, making it one of the leading causes of road fatalities. According to the Metropolitan Manila Development Authority (MMDA), in 2022, motorcycle-related accidents accounted for more than 30\% of road crash incidents in Metro Manila alone, resulting in severe injuries and fatalities and reported a 17.3 percent increase in motorcycle-related road crashes in 2023. Based on the data from its Road Safety Unit, the MMDA said that a total of 26,599 motorcycle-related crashes were recorded in 2022 \cite{MMDA2023}. 

Republic Act No. 10054, also known as the Motorcycle Helmet Act of 2009, mandates that all motorcycle riders and their passengers wear standard protective helmets while on the road. This law aims to reduce head injuries and fatalities by ensuring that helmets meet specific safety standards. Despite the implementation of Republic Act No. 10054, also known as the Motorcycle Helmet Act of 2009, which mandates all motorcycle riders to wear standard protective helmets, many riders continue to violate this law, leading to preventable deaths \cite{Republic2009}. 

A major challenge in enforcing helmet compliance is the reliance on manual monitoring by law enforcement officers, which is often inconsistent and inefficient. Traditional methods such as road checkpoints and manual inspections require significant resources and are prone to human error. Moreover, with the increasing number of motorcyclists on the road, it has become nearly impossible for authorities to monitor helmet compliance effectively. The absence of a scalable and automated monitoring system contributes to the ongoing problem, creating a need for technological solutions that ensure stricter enforcement of traffic laws.With advancements in artificial intelligence (AI) and computer vision, deep learning technologies have emerged as powerful tools for automating helmet compliance detection. Deep learning, a subset of AI, enables machines to process vast amounts of visual data, recognize patterns, and make accurate classifications. Technologies such as YOLO (You Only Look Once), OpenCV, and TensorFlow allow for real-time helmet detection with high precision, making them ideal for traffic monitoring applications. These technologies have been widely implemented in smart surveillance systems for vehicle detection, passenger counting and now, helmet compliance monitoring.

To address the limitations of manual enforcement, this research proposes the development of a Helmet Compliance Detection prototype using computer vision and deep learning algorithms to automatically detect whether motorcycle riders are wearing helmets correctly. The prototype focuses on enhancing helmet compliance monitoring through several key features. It can accurately determine if a rider is properly wearing a helmet on their head and not just carrying it,  it will  also verify if the helmet is securely fastened and correctly positioned.  Helmets can generally be classified into several categories based on their structure and intended use. The prototype will concentrate on the standard motorcycle helmet, which covers the entire head and includes a chin strap and often a visor. This type of helmet offers the most protection and is typically required by law in many regions. The prototype also includes helmet classification by vehicle type, ensuring that riders wear the appropriate helmets corresponding to their vehicles. For example, motorcycle helmets for motorcycles, bicycle helmets for bicycles, and helmets designed for e-bikes for e-bike riders. This helps prevent the use of improper or mismatched helmets, which are flagged as violations to promote stricter adherence to safety standards.
Moreover, the prototype enforces passenger limits by counting riders to ensure no more than two people are on a motorcycle at any time, any overloading is automatically flagged as a violation. Upon detecting any violations,  a red warning is displayed on the system monitor, and the prototype automatically saves short video clips as evidence, supporting authorities in tracking and penalizing repeat offenders. By integrating these features, the prototype aims to improve road safety, assist law enforcement in effectively implementing helmet laws, and ultimately reduce motorcycle-related accidents and fatalities.

\section{Statement of the Problem}

Many motorcycle riders do not follow helmet laws, which can lead to a high risk of accidents, serious injuries, or even death. Traffic officers currently face challenges in manually checking whether motorcycle riders are wearing helmets, as the process is time-consuming and requires significant effort. Since officers cannot monitor every rider, many violations go unnoticed, making the enforcement of helmet laws difficult. Identifying helmet usage under various conditions will be a challenge for the proposed prototype. In poor lighting such as at night or in dark areas the prototype may struggle to clearly identify the rider’s head. Similarly, in adverse weather conditions like fog or heavy rain, recognizing helmets will be difficult. When there are large numbers of motorcycles, it will be hard to check if each rider is wearing a helmet. Because of these challenges, the proposed prototype will need to be tested to ensure it can accurately detect helmets and provide reliable results. It will be evaluated under different conditions such as varied weather and lighting. Its speed and real-time detection performance must also be assessed to ensure it will be reliable in supporting road safety efforts.

\section{Objectives of the Study}
This section outlines the study’s objectives in developing an AI-based helmet detection prototype to improve road safety.

\subsection{General Objective}

The main objective of this study will be to design and develop a Helmet Compliance Detection Using Computer Vision for Safer Roads that will effectively monitor and detect helmet violations among motorcycle riders using Artificial Intelligence (AI), Deep Learning, and Computer Vision. This prototype will aim to provide an accurate and automated solution for identifying non-compliance with helmet regulations, reduce the reliance on manual monitoring, and enhance the enforcement of road safety laws.

\subsection{Specific Objectives}

The specific objectives of this study are as follows:

\begin{enumerate}
    \item Implement deep learning models using YOLO with TensorFlow for object detection, and OpenCV for image and video processing.
    \item Develop an Artificial Intelligence-based prototype integrating the implemented models for helmet detection. 
    \item Evaluate the performance of the designed helmet detection prototype under different conditions such as lighting variations, weather changes, and multiple riders.  
\end{enumerate}
   

\section{Significance of the Study}

This study will focus on applying Artificial Intelligence in traffic law enforcement, particularly in monitoring motorcycle helmet compliance. It will benefit the following stakeholders:

\begin{itemize}
    \item \textbf{Students.} Particularly those studying Computer Science can gain valuable insights into the practical applications of AI in traffic law enforcement. This study serves as a reference for developing intelligent transportation systems and encourages innovative approaches to road safety.
    
    \item \textbf{Motorcycle Riders.} By ensuring helmet compliance, the prototype promotes rider safety, reducing the risk of severe injuries or fatalities. It encourages responsible riding behavior and contributes to safer roads.
    
    \item \textbf{Law Enforcement.} The prototype automates helmet compliance monitoring, reducing manual inspections and improving accuracy. It enhances efficiency, minimizes human error, and provides valuable data for road safety policies.
    
    \item \textbf{Camarines Sur.} The implementation of this prototype can benefit Camarines Sur by improving road safety and reducing motorcycle-related accidents. Local authorities can use this technology to enhance traffic enforcement, ensuring compliance with helmet laws and fostering a safer commuting environment for residents.
    
    \item \textbf{Researcher.} This research establishes a foundation for AI-driven traffic monitoring, enabling further studies in deep learning, object detection, and real-time surveillance, advancing smart city technologies.
    
    \item \textbf{Future Researchers.} The study lays the foundation for further research on AI-driven law enforcement systems, enabling advancements such as database integration and expanded traffic violation detection.
\end{itemize}

\section{Scope and Limitation}

This study will aim to develop and implement an AI-based prototype that uses YOLOv8 with TensorFlow for detecting motorcycles, e-bikes, and bicycles, as well as helmet usage. OpenCV will be used for real-time video and image processing, and the prototype will identify whether riders are wearing helmets properly. In addition, it will count the number of passengers on each vehicle to ensure compliance with road regulations, particularly limiting motorcycle passengers to two. The prototype will be deployed along Nabua Highway. The implementation will involve capturing real-time video through strategically placed surveillance cameras. The captured data will be processed using a Raspberry Pi 4 or NVIDIA Jetson Nano, running the trained YOLOv8 model to detect safety violations.

The prototype’s outputs including flagged violations such as no helmet, improper helmet use, or overloading will be stored as video clips for review by authorities. These outputs will be used to support law enforcement in improving road safety and compliance. However, the prototype has limitations. It will only function effectively in areas covered by surveillance cameras. Its accuracy may decline in low-light or adverse weather conditions such as rain or fog. Recognizing helmet types and differentiating among similar vehicle types (e.g., between bicycles, e-bikes and motorcycles) may introduce errors. The prototype will not be connected directly to enforcement systems during the pilot implementation phase and will initially function as a standalone prototype. Future integrations may include cloud-based databases, vehicle registration systems, and mobile alert features for violations.


\section{Project Dictionary}

To avoid problems in understanding the terms used, the following technical terms are conceptually and operationally defined to provide better understanding. 

\begin{itemize}
	\item \textbf{AI (Artificial Intelligence).} The simulation of human intelligence in machines that enables them to perform tasks such as learning, reasoning, and visual recognition \cite{Russell2021}. In this study, the prototype integrates AI-powered computer vision models to automatically analyze video data, detect helmets, count passengers, and recognize plate numbers without human intervention.

    \item \textbf{Algorithm.} A set of well-defined instructions or rules used to solve a specific problem or perform a computation \cite{Cormen2009}. In this study, the prototype uses machine learning and image processing algorithms to detect helmets, count passengers, and recognize plate numbers from camera feeds. 

    \item \textbf{Accident Prevention.} Encompasses strategies and measures aimed at reducing the occurrence of unintended events that result in injury, death, or property damage. It involves identifying potential hazards, assessing risks, and implementing interventions to mitigate these risks~\cite{HarmsRingdahl2013}. In this study, accident prevention refers to the deployment of artificial intelligence (AI) and computer vision technologies to monitor and analyze real-time data from surveillance prototypes. The goal is to detect and alert authorities about potential accidents or safety violations, thereby enabling timely interventions to prevent incidents.

    \item \textbf{Computer Vision.} A field of artificial intelligence that enables computers and systems to derive meaningful information from digital images, videos, and other visual inputs \cite{Szeliski2010}. In this study, the prototype processes video feeds from cameras to automatically detect helmets, count passengers, and recognize plate numbers without manual intervention.

    \item \textbf{Dataset.} A structured collection of data used to train or evaluate machine learning models. In computer vision, datasets consist of labeled images or videos \cite{Deng2009}. In this study, the prototype utilizes a dataset containing images of motorcycle riders with and without helmets, plate numbers, and various riding conditions to train the object detection model. These datasets can be sourced from public datasets or collected manually for model training and validation.

    \item \textbf{Deep Learning.} A subset of machine learning involving neural networks with multiple layers that learn patterns and representations from large datasets \cite{Goodfellow2016}. In this study, AI models will be used to detect helmets in video using deep neural networks.

    \item \textbf{Helmet.} A protective covering for the head, typically made of a hard material, used as part of safety gear to prevent head injuries \cite{MerriamHelmet}. In this study, it is what the prototype will identify in real-time using computer vision techniques and ensures that the riders wear it properly. 

    \item \textbf{Helmet Compliance.} Wearing of a helmet the right way and following the law when riding a motorcycle \cite{WHO2018}. It helps prevent injuries and deaths in road accidents. In this study, helmet compliance is the main focus. The system uses YOLOv8 to check if riders are wearing helmets properly and to spot those who are not, to help make roads safer using technology.

    \item \textbf{Helmet Detection.} A computer vision task that involves identifying and verifying the presence of a helmet on a person in images or videos \cite{Hayat2022}. In this study, the prototype detects helmets in real-time using computer vision algorithms and determines if they are worn on the head and not held or carried by the riders.

    \item \textbf{Image Processing.} The manipulation of images through computational algorithms to enhance quality or extract useful information \cite{Gonzalez2018}. In this study, image processing techniques will analyze video footage to detect helmets, license plates and passenger counting, ensuring compliance with safety regulations and identifying violations.

    \item \textbf{Law Enforcement.} Refers to the system and practices used by government agencies to ensure public order, uphold laws, and prevent or investigate criminal activities \cite{BritannicaLaw}. In this study, AI-driven surveillance aids authorities by detecting violations, gathering evidence, and enhancing enforcement efficiency through real-time monitoring.

    \item \textbf{Object Detection.} A computer vision technique that identifies and locates objects within an image or video \cite{Tian2019}. In this study, object detection can be used to recognize motorcyclists and determine whether they are wearing helmets by analyzing real-time footage from surveillance cameras or traffic monitoring.

    \item \textbf{Passenger Counting.} The process of counting the number of passengers in a vehicle using sensors or computer vision techniques \cite{Kim2022}. In this study, the prototype employs image processing and object detection to count passengers and compare this number with detected helmets to ensure compliance.

    \item \textbf{Road Safety.} The methods and measures used to prevent road users from being killed or seriously injured, including regulations, infrastructure, and education \cite{OxfordRef}. In this study, road safety includes enforcing helmet laws, using an AI-powered monitoring prototype, improving traffic management, and promoting awareness campaigns to reduce head injuries and fatalities among motorcyclists.

    \item \textbf{Traffic Monitoring.} The systematic observation and recording of vehicular movement and flow on roads, often used to manage congestion and improve traffic systems \cite{BritannicaTraffic}. In this study, traffic monitoring involves using AI-powered cameras and sensors to detect motorcyclists, assess helmet usage, and identify potential violations in real time.

    \item \textbf{YOLO (You Only Look Once).} A deep learning-based object detection model that processes an image in a single pass to detect multiple objects in real time \cite{Redmon2020}. In this study, the prototype employs YOLOv5 or YOLOv8 to efficiently detect helmets on motorcycle riders, identify passengers, and locate the motorcycle’s plate number within video footage.
\end{itemize}

% Then replace \subfloat with \begin{subfigure} in your code:

%=======================================================%
%%%%% Do not delete this part %%%%%%
\clearpage
\printbibliography[heading=subbibintoc, title={\texorpdfstring{\centering}{} Notes}]
\end{refsection}